\documentclass[a4paper, 12pt]{article}
\usepackage{gfsartemisia}
\usepackage[T1]{fontenc}
\usepackage[margin=2cm]{geometry}
\usepackage[dvipsnames]{xcolor}
\usepackage[citestyle=verbose]{biblatex}
\addbibresource{refs.bib}
\title{Coastal Tranzition Zone Modeling}

\begin{document}

\maketitle
    
\begin{enumerate}
\item \textcolor{Maroon}{\textbf{\cite{Maskell_2014}}:}

A finite volume model (FVCOM) was used to investigate the combined influence of storm surge and river flow on floodplain inundation on the basis of idealized estuary test cases. The combined influence of storm surge and river discharge typical of extremes in estuary systems in Britain (up to 2 m and 1500 m$^3$ s$^{-1}$) was found to induce interactions that lead to increases in the nontidal residual elevation of up to 0.35 m. However, the extent of the inundation was found to be mainly controlled by the surge elevation. Exceeding the threshold of the up-estuary channel capacity was found to cause a nonlinear increase in the area of the nontidal inundation for any given peak river discharge, after which the rate of increase in inundation area as the surge height increases declines and is determined by the slope of the floodplain. This threshold is determined by the surge elevation with exception of the highest peak river discharges, where the surge elevation threshold is lowered. It was also found that the extent of the interactions and inundation were highly dependent on the geometry of the estuary and the timing of the surge with respect to peak river discharge, in particular the slope of the floodplain and at such times where the river discharge was similar in magnitude to that of the surge and the tide. After calibration an idealized estuary based in the LISFLOOD-FP code, using a simplified form of the two-dimensional (2D) shallow-water equations, was found to simulate the area of maximum inundation to a similar extent as the FVCOM model (based on the full 2D shallow-water equations) with a much reduced computation time. This paper highlights the potential advantages that simplified 2D inundation models may have for simulating estuarine flooding due to combined surge and river discharge, where surge-river interaction due to momentum exchange is insignificant in determining the flood extent and simplified equations capture the dominant hydrological drivers of coastal inundation.

\end{enumerate}
\end{document}
